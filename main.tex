\documentclass{rmutt-seminar}
\usepackage{amssymb}
\usepackage{enumitem}
\usepackage{amsmath}
\usepackage{diagbox}

\article{RELATION BETWEEN TERMS OF SEQUENCES AND INTEGRAL POWERS OF METALLIC RATIOS}
\author{%
Dr. R. SIVARAMAN
}
\journal{Turkish Journal of Physiotherapy and Rehabilitation}
\articledate{2021}
\articlevolnum{Vol. 32, No. 2}
\articlepages{pp. 1308--1311}
\student{Mr. Kittikun Parinyaprasert}
\studentid{1162109010212}
\advisor{Asst. Prof. Somnuk Srisawat}

\begin{document}

\makeseminartitle

%บทคัดย่อ
\begin{abstract}
	Among several interesting sequences that occur in mathematics, sequences whose successive terms converging to specific 
	numbers called metallic ratios are very special having plenty of applications in branches of science, engineering and 
	nature. In this paper, I will introduce the general sequence corresponding to metallic ratios and obtain interesting 
	relationship between the terms and its integral powers. 
\end{abstract}
%-------------------------------------------

%เนื้อหา

%-------------------------------------------

%หัวข้อ
\section{Introduction}
It is well known that the ratio of successive terms of Fibonacci sequence converges to Golden Ratio. We can
generalize the Fibonacci sequence in a natural way so that the ratio of successive terms converges to specific real 
numbers called Metallic Ratios. In particular, the golden, silver and bronze ratios are special cases of these
metallic ratios. In this paper, I will prove some interesting theorems relating terms of the sequence that are 
defined recursively and integral powers of metallic ratios.

\begin{definition}
Let $ k $ be a natural number. The terms of metallic ratio of order k sequence is defined recursively by 
\begin{equation}\label{recursive metallic}
	 M_{n+2} =kM_{n+1}+M_n 
\end{equation}
for $n \geq 1 $ with initial condition $M_0 = 0, M_1 = 1, M_2 = k$. 

The terms of the metallic ratio of order k sequence as defined in \eqref{recursive metallic} are given by 
\begin{equation}\label{metallic ratio order k sequence}
	 0,1, k, k^2 + 1, k^3 + 2k, k^4 + 3k^2 + k, k^5 +4k^3 +k^2 + 2k, k^6 + 5k^4 + k^3 + 5k^2 + k, \ldots
\end{equation}

\subsection{Metallic ratio of order $k$ }
Through the recurrence relation defined in \eqref{recursive metallic}, We can solve for explicitly. Using the shift operator, the 
recurrence relation yield the quadratic equation $ m^2 - km - 1 = 0 $. The two real roots of this quadratic equation are 
given by $ m = \dfrac{kx \pm \sqrt{k^2 + 4} }{2}$. The positive value among these two roots is defined as the metallic ratio of order 
$ k $ denoted by $ \rho_{k} $. Thus,

 \begin{equation}\label{metallic ratio}
	\rho_{k} = \frac{k+\sqrt{k^2 + 4}}{2}
\end{equation}
Since the sum of two roots is $ k $, the other root is
 \begin{equation}\label{other root metallic ratio}
 	  k - \rho_k = \frac{k - \sqrt{k^2 + 4}}{2}
 \end{equation}

\subsection{Special cases of metallic ratio}
\begin{enumerate}[label=(\roman*)]

\item If $ k = 1$, then from (\ref{metallic ratio})
\begin{equation}\label{golden ratio}
	\rho_{1}=\frac{1+\sqrt{5}}{2}
\end{equation} 
is called the golden ratio.
\item If $ k = 2$, then from (\ref{metallic ratio})
\begin{equation}\label{silver ratio}
	\rho_{2}=1+\sqrt{2}
\end{equation} 
is called the silver ratio.
\item If $ k = 3$, then from (\ref{metallic ratio})
\begin{equation}\label{bronze ratio}
	\rho_{3}=\frac{3+\sqrt{13}}{2}
\end{equation}
 is called the bronze ratio.
\end{enumerate}

\end{definition}
The numbers give by (\ref{golden ratio}),(\ref{silver ratio}) and (\ref{bronze ratio}) form the metallic ratios of first second and third orders.
\pagebreak
\begin{lemma}\label{lemma}
	\textnormal{If} $\rho_{k}= \dfrac{k^2 + 4}{2} $ \textnormal{is the metallic ratio of order} $ k $, \textnormal{then}
	\begin{center}
		$ \rho_{k} - \dfrac{1}{\rho_{k}}=k$ 
	\end{center}
\end{lemma}	
\begin{proof}
	From (\ref{metallic ratio}) and (\ref{other root metallic ratio}) we know that $\rho_{k} = \dfrac{k+\sqrt{k^2+4}}{2}, k - \rho_{k} = \dfrac{k-\sqrt{k^2 + 4}}{2}$ are the two real roots of $ m^2 - km -1 = 0 $. Now taking reciprocal of $ \rho_{k} $ we get 
	\begin{align*}
		\frac{1}{\rho_k} &= \frac{1 }{\dfrac{k+\sqrt{k^2 +4}}{2}} \\
		&=\frac{2}{k + \sqrt{k^2 + 4}} \\
		&=\frac{2}{k + \sqrt{k^2 + 4}} \times \frac{k-\sqrt{k^2 + 4}}{k-\sqrt{k^2 + 4}} \\
		&= \frac{2(k - \sqrt{k^2 + 4})}{k^2-(k^2+4)} \\
		&= \frac{2(k - \sqrt{k^2 + 4})}{k^2 - k^2 -4}\\
		&= \frac{2(k - \sqrt{k^2 + 4})}{-4}\\
		&= \frac{k - \sqrt{k^2 + 4}}{-2}\\
		&= - \left (\dfrac{k-\sqrt{k^2 + 4}}{2} \right)\\
		&= -(k-\rho) \\
		&= \rho_k-k\\
	\end{align*}
Hence $\rho_k-\dfrac{1}{\rho_k}=k$. This completes the proof.
\end{proof}
%-------------------------------------------
\section{Preliminary}
\subsection{Properties of Real number (\cite{MIDC14})}
Let $a, b,$ and $c$ represent real numbers.
\begin{enumerate}
\item \textbf{Closure Properties}

\begin{align*}
a+b \quad &\text{is a real number.}\\
ab \quad &\text{is a real number.}
\end{align*}

\item \textbf{Commutative Properties}

\begin{align*}
a+b &= b+ a\\
	ab &= ba
\end{align*}

\item \textbf{Associative Properties}

\begin{align*}
	( a + b ) + c &= ( a + b ) + c \\
	(ab)c &= a(bc)
\end{align*}

\item \textbf{Identity Properties}

There exist a unique real number $0$ such that
\begin{align*}
	 a + 0 = a. \quad \text{and} \quad 0 + a = a.
\end{align*}

There exist a unique real number $1$ such that
\begin{align*}
	 a \cdot 1 = a. \quad \text{and} \quad 1 \cdot a = a.
\end{align*}

\item \textbf{Inverse Properties}

There exist a unique real number $-a$ such that
\begin{align*}
	 a + (-a) = 0. \quad \text{and} \quad -a + a = 0.
\end{align*}

if $a \neq 0$, there exists a  unique real number $\dfrac{1}{a}$ such that
\begin{align*}
	 a \cdot \dfrac{1}{a} = 1. \quad \text{and} \quad \dfrac{1}{a} \cdot a = 1.
\end{align*}

\item \textbf{Distributive Properties}
\begin{align*}
	a(b+c) &= ab + ac\\
	a(b-c) &= ab - ac
\end{align*}
\end{enumerate}
\begin{theorem}[Fraction Properties \cite{ARE08}]
	\textnormal{For all real number} $a, b, c, d,$ \textnormal{and} $k$ \textnormal{(division by 0 excluded)}:
\begin{enumerate} 
	\item $\dfrac{a}{b} = \dfrac{c}{d} \hspace*{0.5cm} \textnormal{if and only if} \hspace*{0.2cm} ad = bc$.
	\item $\dfrac{ka}{kb} = \dfrac{a}{b}$.
	\item $\dfrac{a}{b} \cdot \dfrac{c}{d} = \dfrac{ac}{bd}$.
	\item $\dfrac{a}{b} \div \dfrac{c}{d} = \dfrac{a}{b} \cdot \dfrac{d}{c}$.
	\item $\dfrac{a}{b} + \dfrac{c}{b} = \dfrac{a + c}{b}$.
	\item $\dfrac{a}{b} - \dfrac{c}{b} = \dfrac{a - c}{b}$.
	\item $\dfrac{a}{b} + \dfrac{c}{d} = \dfrac{ad + bc}{bd}$.
\end{enumerate}
\end{theorem}

\begin{theorem}
\textnormal{\textbf{Properties of Integer Exponents(\cite{ARE08})}}.\\[3 mm] \textnormal{For} $n$ \textnormal{and} $m$ \textnormal{integers and} $a$ \textnormal{and} $b$ \textnormal{real numbers:}
\begin{enumerate}
	\item $a^ma^n=a^{m+n}$.
	\item $(a^n)^m=a^{mn}$.
	\item $(ab)^m=a^mb^m$.
	\item $\bigg(\dfrac{a}{b}\bigg)^m =\dfrac{a^m}{b^m} \quad\quad b\neq0$.
	\item $\dfrac{a^m}{a^n} = \begin{cases} \phantom{-} a^{m-n}& 
			 \\ \dfrac{1}{a^{n-m}} &  \quad \quad a \neq 0 \end{cases}$.
\end{enumerate}
\end{theorem}

\subsubsection*{Special Products (\cite{EP95})}
Let $a$ and $b$ be real numbers, variables, or algebraic expressions.
\begin{enumerate}
	\item \textbf{Sum and Difference of Same Terms} \\
	\hspace*{0.4cm} $(a+b)(a-b) = a^2-b^2$
	\item \textbf{Square of a Binomial} \\
	\hspace*{0.4cm} $(a+b)^2 = a^2 +2ab+b^2$ \\
	\hspace*{0.4cm} $(a-b)^2 = a^2 -2ab+b^2$
	\item \textbf{Cube of Binomial} \\
	\hspace*{0.4cm} $(a+b)^3 = a^3+3a^2b+3ab^2+b^3$ \\
	\hspace*{0.4cm} $(a-b)^3 = a^3-3a^2b+3ab^2-b^3$
\end{enumerate}

\subsection{Quadratic formula}
	If $ ax^2 +bx + c = 0$, then 
\begin{equation*}\label{quadratic formula}
	x= \frac{-b \pm \sqrt{b^2 - 4ac}}{2a}
\end{equation*}
\subsection{Principle of Strong Mathematical Induction}
Let $P(n)$ be a property that is defined for integer $n$, and let $a$ and $b$ be fixed integers with $a \leq b$. Suppose the following two statements are true:
\begin{enumerate}
\item $P(a), P(a+1), \ldots, and P(b)$ are all true. \textbf{(basis step)}
\item For any integer $k \geq b$, if $P(i)$ is true for all integer $i$ from a through $k$ then $P(k+1)$ is true \textbf{(inductive step)}
\end{enumerate}
Then the statement
\begin{center}
for all integer $n \geq a, \quad P(n)$	
\end{center}
is true. (The supposition that $P(i)$ is true for all integers $i$ from $a$ through $k$ is called the \textbf{inductive hypothesis}. Another way to state the inductive hypothesis is to say that $P(a), P(a+1), \ldots, P(k)$ are all true.)

\section{Main Results}
.
\begin{center}	
\subsection*{RELATION WITH RESPECT TO GOLDEN RAITO}
\end{center}
We will consider $ k = 1 $ in \eqref{recursive metallic} and \eqref{metallic ratio order k sequence}. Then the recurrence relation would become 
\begin{equation}\label{k = 1 in reccurence}
	 M_{n+1} = M_{n+1} + M_n.
\end{equation}  
From \eqref{metallic ratio order k sequence} we get Fibonacci sequence whose terms are given by 
\begin{center}
	$0,1,1,2,3,5,8,13,21,34,55,89,144,233,377,610,\ldots $ 
\end{center}

It is well known that (see \cite{R20}) the ratio of successive terms of the Fibonacci sequence approaches the golden ratio
$ \rho_{1} = \varphi = \dfrac{1+\sqrt{5}}{2}$. We will now prove the following theorem.
\begin{theorem}\label{golden ratio theorem}
\textnormal{If} $\bigl\{ M_n \bigr\}$ \textnormal{is the Fibonacci sequence as defined in \eqref{k = 1 in reccurence} then for any positive integer} $ n $ \textnormal{we have} 
\begin{equation}\label{golden ratio recursive}
	\varphi^n + \left(-\frac{1}{\varphi}\right)^n = M_{n-1} + M_{n+1}.  
\end{equation}
\end{theorem}

\begin{proof}
$\varphi^n + \left(-\dfrac{1}{\varphi}\right)^n = M_{n-1} + M_{n+1} $. for all $n \in \mathbb{N}$

\quad Let P$(n):\varphi^n + \left(-\dfrac{1}{\varphi}\right)^n = M_{n-1} + M_{n+1}$. 

\quad (i)\quad Show P($1$) is true 

\quad \quad \quad That is $\varphi - \dfrac{1}{\varphi} = M_{0} + M_{2}$ 

\quad \quad \quad Consider
\begin{align*}
 	\varphi - \dfrac{1}{\varphi} &= \rho_1 - \dfrac{1}{\rho_1}\\
 \end{align*}
 
\quad \quad \quad Since \eqref{lemma} from lemma.\quad $\rho_k-\dfrac{1}{\rho_k}=k$. Thus we obtain $\rho_1-\dfrac{1}{\rho_1}=1$.
 \begin{align*}
 	\varphi - \dfrac{1}{\varphi} &= 1\\
 	&= o + 1\\
 	&= M_{0} + M_{2}
 \end{align*}
 
\quad \quad \quad $\therefore$ P($1$) is true
 
\quad (ii)\quad Assume P($r-1$) and P($r$) is true

\quad \quad \quad That is 
\begin{align*}
	\varphi^{r-1} + \left(-\dfrac{1}{\varphi}\right)^{r-1} &= M_{r-2} + M_{r} \quad, \\
	\varphi^{r} + \left(-\dfrac{1}{\varphi}\right)^{r} &= M_{r-1} + M_{r+1}
\end{align*}

\quad \quad \quad Show P($r+1$) is true
 
\quad \quad \quad That is show 
 
$$ \varphi^{r+1} + \left(-\dfrac{1}{\varphi}\right)^{r+1}= M_{r} + M_{r+2} $$

\quad \quad \quad Consider
\begin{align*}
	M_{r} + M_{r+2} &= M_{r-1} + M_{r-2} + M_{r+1} + M_r \\
	&= M_{r-2} + M_r + M_{r-1} + M_{r+1} \\
	&= \varphi^{r-1} + \left(-\dfrac{1}{\varphi}\right)^{r-1} +\varphi^{r} + \left(-\dfrac{1}{\varphi}\right)^{r} \\
	&= \varphi^{r-1} + \varphi^{r} + \left(-\dfrac{1}{\varphi}\right)^{r-1} + \left(-\dfrac{1}{\varphi}\right)^{r} \\
	&= \varphi^{r+1}(\varphi^{-2} + \varphi^{-1}) + \left(-\frac{1}{\varphi}\right)^{r+1} \left[ \left(-\frac{1}{\varphi}\right)^{-2} + \left(-\frac{1}{\varphi}\right)^{-1}\right] \\
	&= \varphi^{r+1} \left(\frac{1}{\varphi^2} + \frac{1}{\varphi}\right) + \left(-\frac{1}{\varphi}\right)^{r+1}[(-\varphi^2)+(-\varphi)] \\
	&= \varphi^{r+1}\left(\frac{1+\varphi}{\varphi^2}\right)+\left(-\frac{1}{\varphi}\right)^{r+1}(\varphi^2-\varphi)
\end{align*}

\quad \quad \quad Since $\varphi$ is the root of $m^2 -m -1 =0$. Thus $\varphi^2-\varphi-1=0$.
 
\quad \quad \quad  We obtain $\varphi^2=1+\varphi$. and $\varphi^2-\varphi=1$.
 
 \begin{align*}
 M_{r} + M_{r+2} &= \varphi^{r+1}\left(\frac{\varphi^2}{\varphi^2}\right)+\left(-\frac{1}{\varphi}\right)^{r+1}\\
 &= \varphi^{r+1}+\left(-\frac{1}{\varphi}\right)^{r+1}
 \end{align*}	
 
 \quad \quad \quad $\therefore$ P($r+1$) is true.
 
 \quad \quad \quad So by strong mathematical induction principle $\varphi^n + \left(-\dfrac{1}{\varphi}\right)^n = M_{n-1} + M_{n+1} $ 
 
 \quad \quad \quad hold for all $n \in \mathbb{N}$.
\end{proof}

\begin{center}
\subsection*{RELATION WITH RESPECT TO SILVER RATIO}
\end{center}

We will consider $ k = 2 $ in \eqref{recursive metallic} and \eqref{metallic ratio order k sequence}. Then the recurrence relation would become 
\begin{equation}\label{k = 2 in reccurence}
	 M_{n+1} = 2M_{n+1} + M_n.
\end{equation}  
From \eqref{metallic ratio order k sequence} we get Fibonacci sequence whose terms are given by 
\begin{center}
	$0,1,2,5,12,29,70,169,408,985,2378,\ldots $ 
\end{center}

It can be show that (see \cite{R20}) the ratio of successive terms of this sequence approaches the silver ratio
$ \rho_{2} = \lambda = 1 + \sqrt{2} $. We will now prove the following theorem.
\begin{theorem}\label{silver ratio theorem}
\textnormal{If} $\bigl\{ M_n \bigr\}$ \textnormal{is the Fibonacci sequence as defined in \eqref{k = 2 in reccurence} in reccurence then for any positive integer} $ n $ \textnormal{we have} 
\begin{equation}\label{silver ratio recursive}
	 \lambda^n + \left(-\frac{1}{\lambda}\right)^n = M_{n-1} + M_{n+1}.  
\end{equation}
\end{theorem}

\begin{proof}
$\lambda^n + \left(-\dfrac{1}{\lambda}\right)^n = M_{n-1} + M_{n+1}$. for all $n \in \mathbb{N}$

\quad Let P$(n):\lambda^n + \left(-\dfrac{1}{\lambda}\right)^n = M_{n-1} + M_{n+1}$. 

\quad (i)\quad Show P($1$) is true 

\quad \quad \quad That is $\lambda - \dfrac{1}{\lambda} = M_{0} + M_{2}$ 

\quad \quad \quad Consider
\begin{align*}
 	\lambda - \dfrac{1}{\lambda} &= \rho_2 - \dfrac{1}{\rho_2}\\
 \end{align*}
 
\quad \quad \quad Since \eqref{lemma} from lemma.\quad $\rho_k-\dfrac{1}{\rho_k}=k$. Thus we obtain $\rho_2-\dfrac{1}{\rho_2}=2$.
 \begin{align*}
 	\lambda - \dfrac{1}{\lambda} &= 2\\
 	&= o + 2\\
 	&= M_{0} + M_{2}
 \end{align*}
 
\quad \quad \quad $\therefore$ P($1$) is true
 
\quad (ii)\quad Assume P($r-1$) and P($r$) is true

\quad \quad \quad That is 
\begin{align*}
	\lambda^{r-1} + \left(-\dfrac{1}{\lambda}\right)^{r-1} &= M_{r-2} + M_{r} \quad, \\
	\lambda^{r} + \left(-\dfrac{1}{\lambda}\right)^{r} &= M_{r-1} + M_{r+1}
\end{align*}

\quad \quad \quad Show P($r+1$) is true
 
\quad \quad \quad That is show 
 
$$ \lambda^{r+1} + \left(-\dfrac{1}{\lambda}\right)^{r+1}= M_{r} + M_{r+2} $$

\quad \quad \quad Consider
\begin{align*}
	M_{r} + M_{r+2} &= 2M_{r-1} + M_{r-2} + 2M_{r+1} + M_r \\
	&= M_{r-2} + M_r + 2M_{r-1} + 2M_{r+1} \\ 
	&= M_{r-2} + M_r + 2(M_{r-1} + M_{r+1}) \\
	&= \lambda^{r-1} + \left(-\dfrac{1}{\lambda}\right)^{r-1} +2\left[\lambda^{r} + \left(-\dfrac{1}{\lambda}\right)^{r}\right] \\
	&= \lambda^{r-1} + \left(-\dfrac{1}{\lambda}\right)^{r-1} +2\lambda^{r} + 2\left(-\dfrac{1}{\lambda}\right)^{r}\ \\
	&= \lambda^{r-1}+2\lambda^{r}  + \left(-\dfrac{1}{\lambda}\right)^{r-1} + 2\left(-\dfrac{1}{\lambda}\right)^{r}\ \\
	&= \lambda^{r+1}(\lambda^{-2} + 2\lambda^{-1}) + \left(-\frac{1}{\lambda}\right)^{r+1} \left[ \left(-\frac{1}{\lambda}\right)^{-2} + 2\left(-\frac{1}{\lambda}\right)^{-1}\right] \\
	&= \lambda^{r+1} \left(\frac{1}{\lambda^2} + \frac{2}{\lambda}\right) + \left(-\frac{1}{\lambda}\right)^{r+1}[(-\lambda^2)+2(-\lambda)] \\
	&= \lambda^{r+1}\left(\frac{1+2\lambda}{\lambda^2}\right)+\left(-\frac{1}{\lambda}\right)^{r+1}(\lambda^2-2\lambda)
\end{align*}

\quad \quad \quad Since $\lambda$ is the root of $m^2 -2m -1 =0$. Thus $\lambda^2-2\lambda-1=0$.
 
\quad \quad \quad  We obtain $\lambda^2=1+2\lambda$. and $\lambda^2-2\lambda=1$.
 
 \begin{align*}
 M_{r} + M_{r+2} &= \lambda^{r+1}\left(\frac{\lambda^2}{\lambda^2}\right)+\left(-\frac{1}{\lambda}\right)^{r+1} \\
 &= \lambda^{r+1}+\left(-\frac{1}{\lambda}\right)^{r+1}
 \end{align*}	
 
 \quad \quad \quad $\therefore$ P($r+1$) is true.
 
 \quad \quad \quad So by strong mathematical induction principle $\lambda^n + \left(-\dfrac{1}{\lambda}\right)^n = M_{n-1} + M_{n+1}$. 
 
 \quad \quad \quad hold for all $n \in \mathbb{N}$.
\end{proof}

\begin{center}
\subsection*{RELATION WITH RESPECT TO BRONZE RATIO}
\end{center}
We will consider $ k = 3 $ in \eqref{recursive metallic} and \eqref{metallic ratio order k sequence}. Then the recurrence relation would become 
\begin{equation}\label{k = 3 in reccurence}
	 M_{n+1} = 3M_{n+1} + M_n.
\end{equation}  
From \eqref{metallic ratio order k sequence} we get Fibonacci sequence whose terms are given by 
\begin{center}
	$0,1,3,10,33,109,360,1189,3927,12970,\ldots $ 
\end{center}

It can be show that (see \cite{R20}) the ratio of successive terms of this sequence approaches the bronze ratio
$ \rho_{3} = \mu = \dfrac{3 + \sqrt{13}}{2}$. We will now prove the following theorem.
\begin{theorem}\label{bronze ratio theorem}
\textnormal{If} $\bigl\{ M_n \bigr\}$ \textnormal{is the Fibonacci sequence as defined in \eqref{k = 3 in reccurence} then for any positive integer} $ n $ \textnormal{we have} 
\begin{equation}\label{bronze ratio recursive}
	 \mu^n + \left(-\frac{1}{\mu}\right)^n = M_{n-1} + M_{n+1}. 
\end{equation}
\end{theorem}

\begin{proof}
$\mu^n + \left(-\dfrac{1}{\mu}\right)^n = M_{n-1} + M_{n+1} $. for all $n \in \mathbb{N}$

\quad Let P($n$) : $\mu^n + \left(-\dfrac{1}{\mu}\right)^n = M_{n-1} + M_{n+1} $. 

\quad (i)\quad Show P($1$) is true 

\quad \quad \quad That is $\mu -\dfrac{1}{\mu} = M_{0} + M_{2} $ 

\quad \quad \quad Consider
\begin{align*}
 	\mu - \dfrac{1}{\mu} &= \rho_3 - \dfrac{1}{\rho_3}\\
 \end{align*}
 
\quad \quad \quad Since \eqref{lemma} from lemma.\quad $\rho_k-\dfrac{1}{\rho_k}=k$. Thus we obtain $\rho_3-\dfrac{1}{\rho_3}=3$.
 \begin{align*}
 	\mu - \dfrac{1}{\mu} &= 3\\
 	&= o + 3\\
 	&= M_{0} + M_{2}
 \end{align*}
 
\quad \quad \quad $\therefore$ P($1$) is true
 
\quad (ii)\quad Assume P($r-1$) and P($r$) is true

\quad \quad \quad That is 
\begin{align*}
	\mu^{r-1} + \left(-\dfrac{1}{\mu}\right)^{r-1} &= M_{r-2} + M_{r} \quad, \\
	\mu^{r} + \left(-\dfrac{1}{\mu}\right)^{r} &= M_{r-1} + M_{r+1}
\end{align*}

\quad \quad \quad Show P($r+1$) is true
 
\quad \quad \quad That is show 
 
$$ \mu^{r+1} + \left(-\dfrac{1}{\mu}\right)^{r+1}= M_{r} + M_{r+2} $$

\quad \quad \quad Consider
\begin{align*}
	M_{r} + M_{r+2} &= 3M_{r-1} + M_{r-2} + 3M_{r+1} + M_r \\
	&= M_{r-2} + M_r + 3M_{r-1} + 3M_{r+1} \\ 
	&= M_{r-2} + M_r + 3(M_{r-1} + M_{r+1}) \\
	&= \mu^{r-1} + \left(-\dfrac{1}{\mu}\right)^{r-1} +3\left[\mu^{r} + \left(-\dfrac{1}{\mu}\right)^{r}\right] \\
	&= \mu^{r-1} + \left(-\dfrac{1}{\mu}\right)^{r-1} +3\mu^{r} + 3\left(-\dfrac{1}{\mu}\right)^{r}\ \\
	&= \mu^{r-1}+3\mu^{r}  + \left(-\dfrac{1}{\mu}\right)^{r-1} + 3\left(-\dfrac{1}{\mu}\right)^{r}\ \\
	&= \mu^{r+1}(\mu^{-2} + 3\mu^{-1}) + \left(-\frac{1}{\mu}\right)^{r+1} \left[ \left(-\frac{1}{\mu}\right)^{-2} + 3\left(-\frac{1}{\mu}\right)^{-1}\right] \\
	&= \mu^{r+1} \left(\frac{1}{\mu^2} + \frac{3}{\mu}\right) + \left(-\frac{1}{\mu}\right)^{r+1}[(-\mu^2)+3(-\mu)] \\
	&= \mu^{r+1}\left(\frac{1+3\mu}{\mu^2}\right)+\left(-\frac{1}{\mu}\right)^{r+1}(\mu^2-3\mu)
\end{align*}

\quad \quad \quad Since $\mu$ is the root of $m^2 -3m -1 =0$. Thus $\mu^2-3\mu-1=0$.
 
\quad \quad \quad  We obtain $\mu^2=1+3\mu$. and $\mu^2-3\mu=1$.
 
 \begin{align*}
 M_{r} + M_{r+2} &= \mu^{r+1}\left(\frac{\mu^2}{\mu^2}\right)+\left(-\frac{1}{\mu}\right)^{r+1} \\
 &= \mu^{r+1}+\left(-\frac{1}{\mu}\right)^{r+1}
 \end{align*}	
 
 \quad \quad \quad $\therefore$ P($r+1$) is true.
 
 \quad \quad \quad So by strong mathematical induction principle $\mu^n + \left(-\dfrac{1}{\mu}\right)^n = M_{n-1} + M_{n+1}$. 
 
 \quad \quad \quad hold for all $n \in \mathbb{N}$.
\end{proof}

\begin{center}
\subsection*{RELATION WITH RESPECT TO GENERAL METALLIC RATIO}
\end{center}

In this section, I will obtain a general result connecting terms of the metallic ratio sequence defined in \eqref{recursive metallic} and 
\eqref{metallic ratio order k sequence}. Then the recurrence relation of general metallic ratio sequence is given by
\begin{equation*}
	 M_{n+2} =kM_{n+1}+M_n.
\end{equation*}  
From \eqref{metallic ratio order k sequence} we get a sequence whose terms are given by 
\begin{center}
	$0,1,k,k^2 + 1,k^3 + 2k,k^4 + 3k^2 + k, k^5 + 4k^3 + k^2 + 2k, k^6 + 5k^4 +k^3 + 5k^2 + k,\ldots$ 
\end{center}

It can be shown that (see \cite{R20}) the ratio of successive terms of this sequence approaches the metallic ratio of order $k$ given by $\rho_k = \dfrac{k+\sqrt{k^2+4}}{2}$. We 
will now prove the following theorem.
\begin{theorem}\label{metallic ratio theorem}
\textnormal{If} $\bigl\{ M_n \bigr\}$ \textnormal{is the Fibonacci sequence as defined in \eqref{recursive metallic} then for any positive integer} $ n $ \textnormal{we have} 
\begin{equation}\label{metallic ratio recursive}
	 (\rho_k )^n + \left(-\frac{1}{\rho_k }\right)^n = M_{n-1} + M_{n+1}. 
\end{equation}
\end{theorem}

\begin{proof}
$(\rho_k )^n + \left(-\dfrac{1}{\rho_k }\right)^n = M_{n-1} + M_{n+1} $. for all $n \in \mathbb{N}$

\quad Let P($n$) : $(\rho_k) ^n + \left(-\dfrac{1}{\rho_k }\right)^n = M_{n-1} + M_{n+1} $. 

\quad (i)\quad Show P($1$) is true 

\quad \quad \quad That is $\rho_k -\dfrac{1}{\rho_k} = M_{0} + M_{2} $ 
 
\quad \quad \quad Since \eqref{lemma} from lemma. We obtain \quad $\rho_k-\dfrac{1}{\rho_k}=k$.

\quad \quad \quad Consider

 \begin{align*}
 	\rho_k - \dfrac{1}{\rho_k} &= k\\
 	&= o + k\\
 	&= M_{0} + M_{2}
 \end{align*}
 
\quad \quad \quad $\therefore$ P($1$) is true
 
\quad (ii)\quad Assume P($r-1$) and P($r$) is true

\quad \quad \quad That is 
\begin{align*}
	(\rho_k)^{r-1} + \left(-\dfrac{1}{\rho_k}\right)^{r-1} &= M_{r-2} + M_{r} \quad, \\
	(\rho_k)^{r} + \left(-\dfrac{1}{\rho_k}\right)^{r} &= M_{r-1} + M_{r+1}
\end{align*}

\quad \quad \quad Show P($r+1$) is true
 
\quad \quad \quad That is show 
 
$$ (\rho_k)^{r+1} + \left(-\dfrac{1}{\rho_k}\right)^{r+1}= M_{r} + M_{r+2} $$

\quad \quad \quad Consider
\begin{align*}
	M_{r} + M_{r+2} &= kM_{r-1} + M_{r-2} + kM_{r+1} + M_r \\
	&= M_{r-2} + M_r + kM_{r-1} + kM_{r+1} \\ 
	&= M_{r-2} + M_r + k(M_{r-1} + M_{r+1}) \\
	&= (\rho_k)^{r-1} + \left(-\dfrac{1}{\rho_k}\right)^{r-1} +k\left[(\rho_k)^{r} + \left(-\dfrac{1}{\rho_k}\right)^{r}\right] \\
	&= (\rho_k)^{r-1} + \left(-\dfrac{1}{\rho_k}\right)^{r-1} +k(\rho_k)^{r} + k\left(-\dfrac{1}{\rho_k}\right)^{r}\ \\
	&= (\rho_k)^{r-1}+k(\rho_k)^{r}  + \left(-\dfrac{1}{\rho_k}\right)^{r-1} + k\left(-\dfrac{1}{\rho_k}\right)^{r}\ \\
	&= (\rho_k)^{r+1}((\rho_k)^{-2} + k(\rho_k)^{-1}) + \left(-\frac{1}{\rho_k}\right)^{r+1} \left[ \left(-\frac{1}{\rho_k}\right)^{-2} + k\left(-\frac{1}{\rho_k}\right)^{-1}\right] \\
	&= (\rho_k)^{r+1} \left(\frac{1}{(\rho_k)^2} + \frac{k}{\rho_k}\right) + \left(-\frac{1}{\rho_k}\right)^{r+1}[(-\rho_k)^2+k(-\rho_k)] \\
	&= (\rho_k)^{r+1}\left(\frac{1+k\rho_k}{(\rho_k)^2}\right)+\left(-\frac{1}{\rho_k}\right)^{r+1}((\rho_k)^2-k\rho_k)
\end{align*}

\quad \quad \quad Since $\rho_k$ is the root of $m^2 -km -1 =0$. 
 
\quad \quad \quad  We obtain $(\rho_k)^2=1+k\rho_k$. and $(\rho_k)^2-k\rho_k=1$.
 
 \begin{align*}
 M_{r} + M_{r+2} &= (\rho_k)^{r+1}\left(\dfrac{(\rho_k)^2}{(\rho_k)^2}\right)+\left(-\frac{1}{\rho_k}\right)^{r+1}\\
 &= (\rho_k)^{r+1}+\left(-\frac{1}{\rho_k}\right)^{r+1}
 \end{align*}	
 
 \quad \quad \quad $\therefore$ P($r+1$) is true.
 
 \quad \quad \quad So by strong mathematical induction principle $(\rho_k )^n + \left(-\frac{1}{\rho_k }\right)^n = M_{n-1} +$
 
 \quad \quad \quad $ M_{n+1}$. hold for all $n \in \mathbb{N}$.
\end{proof}

\section{Conclusion}

Introducing golden, silver and bronze ratios through general metallic ratio sequence as defined in \eqref{recursive metallic} and \eqref{metallic ratio order k sequence} 
I had established three interesting theorems \ref{golden ratio theorem}, \ref{silver ratio theorem} and \ref{bronze ratio theorem} respectively. In section of relation with respect to the general metallic ratio, I had 
established the same result for general metallic ratio of order $k$. It is amusing to notice that all metallic ratios 
satisfy similar relation regarding the integral powers and alternate terms of the sequence. This result may pave 
way for proving other similar results regarding metallic ratios.

%--------------------------

\printbibliography

%--------------------------
	
\end{document}