\documentclass{rmutt-seminar}
\usepackage{amssymb}
\usepackage{enumitem}
\usepackage{amsmath}
\usepackage{diagbox}

\article{RELATION BETWEEN TERMS OF SEQUENCES AND INTEGRAL POWERS OF METALLIC RATIOS}
\author{%
Dr. R. SIVARAMAN
}
\journal{Turkish Journal of Physiotherapy and Rehabilitation}
\articledate{2021}
\articlevolnum{Vol. 32, No. 2}
\articlepages{pp. 1308--1311}
\student{Mr. Kittikun Parinyaprasert}
\studentid{1162109010212}
\advisor{Asst. Prof. Somnuk Srisawat}

\begin{document}

\makeseminartitle

%บทคัดย่อ
\begin{abstract}
	Among several interesting sequences that occur in mathematics, sequences whose successive terms converging to specific 
	numbers called metallic ratios are very special having plenty of applications in branches of science, engineering and 
	nature. In this paper, I will introduce the general sequence corresponding to metallic ratios and obtain interesting 
	relationship between the terms and its integral powers. 
\end{abstract}
%-------------------------------------------

%เนื้อหา

%-------------------------------------------

%หัวข้อ
\section{Introduction}
It is well known that the ratio of successive terms of Fibonacci sequence converges to Golden Ratio. We can
generalize the Fibonacci sequence in a natural way so that the ratio of successive terms converges to specific real 
numbers called Metallic Ratios. In particular, the golden, silver and bronze ratios are special cases of these
metallic ratios. In this paper, I will prove some interesting theorems relating terms of the sequence that are 
defined recursively and integral powers of metallic ratios.

\begin{definition}
Let $ k $ be a natural number. The terms of metallic ratio of order $k$ sequence is defined recursively by 
\begin{equation}\label{recursive metallic}
	M_{n+2} =kM_{n+1}+M_n,\textnormal{ for } n \geq 1 
\end{equation}
with initial conditions $M_0 = 0, M_1 = 1, M_2 = k$. 
\newline 

The terms of the metallic ratio of order $k$ sequence as defined in equation \eqref{recursive metallic} are given by 
\begin{equation}\label{metallic ratio order k sequence}
	 0,1, k, k^2 + 1, k^3 + 2k, k^4 + 3k^2 + k, k^5 +4k^3 +k^2 + 2k, k^6 + 5k^4 + k^3 + 5k^2 + k, \ldots
\end{equation}

\subsection{Metallic ratio of order $ k $}
Through the recurrence relation defined in equation \eqref{recursive metallic}, we can solve for explicitly. Using the shift operator, the 
recurrence relation yield the quadratic equation $ m^2 - km - 1 = 0 $. The two real roots of this quadratic equation are 
given by $ m = \dfrac{k \pm \sqrt{k^2 + 4} }{2}$. The positive value among these two roots is defined as the metallic ratio of order 
$ k $ denoted by $ \rho_{k} $. Thus,

 \begin{equation}\label{metallic ratio}
	\rho_{k} = \frac{k+\sqrt{k^2 + 4}}{2}.
\end{equation}
Since the sum of two roots is $ k $, the other root is
 \begin{equation}\label{other root metallic ratio}
 	  k - \rho_k = \frac{k - \sqrt{k^2 + 4}}{2}.
 \end{equation}

\subsection{Special cases of metallic ratio}
\begin{enumerate}[label=(\roman*)]

\item If $ k = 1$, then from equation (\ref{metallic ratio})
\begin{equation}\label{golden ratio}
	\rho_{1}=\frac{1+\sqrt{5}}{2}
\end{equation} 
is called the golden ratio.
\item If $ k = 2$, then from equation  (\ref{metallic ratio})
\begin{equation}\label{silver ratio}
	\rho_{2}=1+\sqrt{2}
\end{equation} 
is called the silver ratio.
\item If $ k = 3$, then from equation (\ref{metallic ratio})
\begin{equation}\label{bronze ratio}
	\rho_{3}=\frac{3+\sqrt{13}}{2}
\end{equation}
 is called the bronze ratio.
\end{enumerate}

\end{definition}
The numbers give by equation (\ref{golden ratio}),(\ref{silver ratio}) and (\ref{bronze ratio}) form the metallic ratios of first second and third orders, respectively.
\begin{lemma}\label{lemma}
	If $\rho_{k}= \dfrac{k^2 + 4}{2} $ is the metallic ratio of order $ k $, then
	\begin{center}
		$ \rho_{k} - \dfrac{1}{\rho_{k}}=k$ 
	\end{center}
\end{lemma}	
\begin{proof}
	From equation (\ref{metallic ratio}) and (\ref{other root metallic ratio}), we know that $\rho_{k} = \dfrac{k+\sqrt{k^2+4}}{2}$ and $k - \rho_{k} = \dfrac{k-\sqrt{k^2 + 4}}{2}$ are the two real roots of $ m^2 - km -1 = 0 $. We have 
	\begin{align*}
		\frac{1}{\rho_k} &= \frac{1 }{\dfrac{k+\sqrt{k^2 +4}}{2}} \\
		&=\frac{2}{k + \sqrt{k^2 + 4}} \\
		&=\frac{2(k-\sqrt{k^2 + 4)}}{(k + \sqrt{k^2 + 4})(k-\sqrt{k^2 + 4)} } \\
		&= \frac{2(k - \sqrt{k^2 + 4})}{k^2-(k^2+4)} \\
		&= \frac{2(k - \sqrt{k^2 + 4})}{k^2 - k^2 -4}\\
		&= \frac{2(k - \sqrt{k^2 + 4})}{-4}\\
		&= \frac{k - \sqrt{k^2 + 4}}{-2}\\
		&= - \dfrac{k-\sqrt{k^2 + 4}}{2} \\
		&= -(k-\rho_k) \\
		&= \rho_k-k,
	\end{align*}
hence 

\begin{align*}
	\frac{1}{\rho_k}+k&=\rho_k \\
	k &= \rho_k - \frac{1}{k} 
\end{align*}

\end{proof}
%-------------------------------------------
\section{Preliminary}
\subsection*{Properties of real number (\cite{MIDC14}) }

\indent Let $a, b,$ and $c$ represent real numbers.
\begin{enumerate}
\item Closure Properties
\begin{align*}
a+b \quad &\text{is a real number.}\\
ab \quad &\text{is a real number.}
\end{align*}
\item Commutative Properties
\begin{align*}
a+b &= b+ a.\\
	ab &= ba.
\end{align*}

\item Associative Properties
\begin{align*}
	( a + b ) + c &= ( a + b ) + c. \\
	(ab)c &= a(bc).
\end{align*}

\item Identity Properties

There exist a unique real number $0$ such that
\begin{align*}
	 a + 0 = a \quad \text{and} \quad 0 + a = a.
\end{align*}

There exist a unique real number $1$ such that
\begin{align*}
	 a \cdot 1 = a \quad \text{and} \quad 1 \cdot a = a.
\end{align*}

\item Inverse Properties

There exist a unique real number $-a$ such that
\begin{align*}
	 a + (-a) = 0 \quad \text{and} \quad -a + a = 0.
\end{align*}

if $a \neq 0$, there exists a  unique real numbers $\dfrac{1}{a}$ such that
\begin{align*}
	 a \cdot \dfrac{1}{a} = 1 \quad \text{and} \quad \dfrac{1}{a} \cdot a = 1.
\end{align*}

\item Distributive Properties
\begin{align*}
	a(b+c) &= ab + ac\\
	a(b-c) &= ab - ac
\end{align*}
\end{enumerate}
\noindent\textbf{Properties of fractions(\cite{ARE08})}
\newline

Let $a, b, c$ and $d$ be real numbers, variables or algebraic expressions such that $b \neq 0$ and $d \neq 0$.
\begin{enumerate} 
	\item Equivalent fractions:
	\begin{equation*}
		\frac{a}{b} = \frac{c}{d} \hspace*{0.5cm} \textnormal{if and only if} \hspace*{0.2cm} ad = bc.
	\end{equation*}
	\item Rules of signs:
	\begin{equation*}
		-\frac{a}{b} = \frac{-a}{b} = \frac{a}{-b}\quad \textnormal{ and } \quad \frac{-a}{-b} = \frac{a}{b}.
	\end{equation*}
	\item Generate equivalent fractions:
	\begin{equation*}
		\frac{a}{b} = \frac{ac}{bc}, \quad c \neq 0.
	\end{equation*}
	\item Add or subtract with like denominators: 
	\begin{equation*}
		\frac{a}{b} \pm \frac{c}{b} = \frac{a \pm c}{b}.
	\end{equation*}	
	\item Add or subtract with unlike denominators:
	\begin{equation*}
		\frac{a}{b} \cdot \frac{c}{d} = \frac{ac}{bd}.
	\end{equation*}
	\item Divide fractions:
	\begin{equation*}
		\frac{a}{b} \div \frac{c}{d} = \frac{a}{b} \cdot \frac{d}{c},\quad c \neq 0.
	\end{equation*}
\end{enumerate}

\subsection*{Properties of integer exponents (\cite{ARE08})} For $n$ and $m$ be integers and $a$ and $b$ are real numbers.
\begin{enumerate}
	\item $a^ma^n=a^{m+n}$.
	\item $(a^n)^m=a^{mn}$.
	\item $(ab)^m=a^mb^m$.
	\item $\bigg(\dfrac{a}{b}\bigg)^m =\dfrac{a^m}{b^m},b\neq0$.
	\item $\dfrac{a^m}{a^n} = \begin{cases} \phantom{-} a^{m-n}& 
			 \\ \dfrac{1}{a^{n-m}},a \neq 0. \end{cases}$\\
\end{enumerate}

\subsubsection*{Special products (\cite{EP95})}
Let $a$ and $b$ be real numbers, variables, or algebraic expressions.
\begin{enumerate}
	\item Sum and difference of same terms 
	 $$(a+b)(a-b) = a^2-b^2.$$
	\item Square of a Binomial 
	 $$(a+b)^2 = a^2 +2ab+b^2.$$ 
	 $$(a-b)^2 = a^2 -2ab+b^2.$$
	\item Cube of Binomial 
	 $$(a+b)^3 = a^3+3a^2b+3ab^2+b^3.$$ 
	 $$(a-b)^3 = a^3-3a^2b+3ab^2-b^3.$$
\end{enumerate}

\subsection*{Quadratic formula}
\begin{theorem}[\cite{DC95}]
	If $ p(x) = ax^2 +bx + c = 0, a \neq 0 $ and $ 0 \leq b^2 - 4ac $. Then
	 
	The two real roots of  p are 
\begin{equation*}\label{quadratic formula}
	x= \frac{-b \pm \sqrt{b^2 - 4ac}}{2a}.
\end{equation*}
\end{theorem}
\subsection*{Principle of Strong Mathematical Induction}
\begin{theorem}[\cite{C10}] 
	To prove that $P(n)$ is true for all positive integer $n$, Where $P(n)$ is a propositional function, we complete two steps:
	\newline
	\indent Basis step : We verify that the proposition $P(1)$ is true.
	
	Inductive step: We show that the conditional statement  $[ P(1) \land P(2) \land \ldots ,P(k)]  \to P(k+1)$ is true for all positive integers $k$.
\end{theorem}
\section{Main Results}
\par
We will consider $ k = 1 $ in equation \eqref{recursive metallic} and \eqref{metallic ratio order k sequence}. Then the recurrence relation would become 
\begin{equation}\label{k = 1 in reccurence}
	 M_{n+2} = M_{n+1} + M_n, \textnormal{ for } n \geq 1
\end{equation}  
where $M_0=0, M_1=1, M_2=1.$
\newline

\noindent From equation \eqref{metallic ratio order k sequence} we get Fibonacci sequence whose terms are given by 
\begin{center}
	$0,1,1,2,3,5,8,13,21,34, 55,89,144,233,377,610,\ldots $ 
\end{center}

It is well known that (see \cite{R20}) the ratio of successive terms of the Fibonacci sequence approaches the golden ratio
$ \rho_{1} = \varphi = \dfrac{1+\sqrt{5}}{2}$. We will now prove the following theorem.
\begin{theorem}\label{golden ratio theorem}
If $\bigl\{ M_n \bigr\}$ is the Fibonacci sequence as defined on equation \eqref{k = 1 in reccurence}, then for any positive integer $ n $, we have
\begin{equation}\label{golden ratio recursive}
	\varphi^n + \left(-\frac{1}{\varphi}\right)^n = M_{n-1} + M_{n+1}.  
\end{equation}
\end{theorem}

\begin{proof} We prove by using principle of strong  mathematical induction on $n \in \mathbb{N}$. 

\quad Let $P(n):\varphi^n + \left(-\dfrac{1}{\varphi}\right)^n = M_{n-1} + M_{n+1}$. 
\begin{enumerate}[label=(\roman*),leftmargin = 1.5cm]
	\item To show  that $P(1)$ is true, that is

 $$\varphi - \dfrac{1}{\varphi} = M_{0} + M_{2}.$$ 
 
 From Lemma \ref{lemma} and $k=1$, we have $\rho_1 - \dfrac{1}{\rho_1} =1 $.
 
 Since $ \varphi = \rho_1$ and $M_0=0, M_2=1 $then 
  \begin{align*}
 	\varphi - \dfrac{1}{\varphi} &= 1\\
 	&= 0 + 1\\
 	&= M_{0} + M_{2}.
 \end{align*}
%  and
 
%  \begin{align*}
%  	\varphi + \frac{1}{\varphi^2} &= \left(\frac{1+\sqrt{5}}{2}\right)^2 + \frac{1}{\left(\dfrac{1+\sqrt{5}}{2}\right)^2}\\
%  	&= \frac{(1+\sqrt{5})^2}{2^2} + \frac{1}{\dfrac{(1+\sqrt{5})^2}{2^2}}\\
%  	&= \frac{1+2\sqrt{5}+(\sqrt{5})^2}{4} + \frac{1}{\dfrac{1+2\sqrt{5}+(\sqrt{5})^2}{4}}\\
%  	&= \frac{1+2\sqrt{5}+5}{4} + \frac{1}{\dfrac{1+2\sqrt{5}+5}{4}}\\
%  	&= \frac{6+2\sqrt{5}}{4} + \frac{1}{\dfrac{6+2\sqrt{5}}{4}}\\
%  	&= \frac{2(3+\sqrt{5})}{4} + \frac{1}{\dfrac{2(3+\sqrt{5})}{4}}\\
%  	&= \frac{3+\sqrt{5}}{2} + \frac{1}{\dfrac{3+\sqrt{5}}{2}}\\
%  	&= \frac{3+\sqrt{5}}{2} + \frac{2}{3+\sqrt{5}}\\
%  	&= \frac{3^2+2(3)\sqrt{5}+(\sqrt{5})^2}{2(3+\sqrt{5})} + \frac{4}{2(3+\sqrt{5})}\\
%  	&= \frac{9+6\sqrt{5}+5}{2(3+\sqrt{5})} + \frac{4}{2(3+\sqrt{5})}\\
%  	&= \frac{9+6\sqrt{5}+5+4}{2(3+\sqrt{5})} \\
%  	&= \frac{18+6\sqrt{5}}{2(3+\sqrt{5})} \\
%  	&= \frac{6(3+\sqrt{5})}{2(3+\sqrt{5})} \\
%  	&= 3 \\
%  	&= 1 + 2 \\
%  	&= M_1 + M_3,
%  \end{align*}
 
 
 Thus  $P(1)$ is true.
 
 \item For $r \geq 1$. We assume that $P(r-1)$ and $P(r)$ are true, that is
 \begin{equation*}
	\varphi^{r-1} + \left(-\dfrac{1}{\varphi}\right)^{r-1} = M_{r-2} + M_{r},  
\end{equation*}
and 
\begin{equation*}
	\varphi^{r} + \left(-\dfrac{1}{\varphi}\right)^{r} = M_{r-1} + M_{r+1},
\end{equation*}

To show that $P(r+1)$ is true, that is
$$ \varphi^{r+1} + \left(-\dfrac{1}{\varphi}\right)^{r+1}= M_{r} + M_{r+2}. $$  
 Consider
 \begin{align*}
	M_{r} + M_{r+2} &= M_{r-1} + M_{r-2} + M_{r+1} + M_r \\
	&= M_{r-2} + M_r + M_{r-1} + M_{r+1} \\
	&= \varphi^{r-1} + \left(-\dfrac{1}{\varphi}\right)^{r-1} +\varphi^{r} + \left(-\dfrac{1}{\varphi}\right)^{r} \\
	&= \varphi^{r-1} + \varphi^{r} + \left(-\dfrac{1}{\varphi}\right)^{r-1} + \left(-\dfrac{1}{\varphi}\right)^{r} \\
%	&= \varphi^{r+1}\varphi^{-2} + \varphi^{r+1}\varphi^{-1} + \left(-\dfrac{1}{\varphi}\right)^{r+1}\left(-\dfrac{1}{\varphi}\right)^{-2} + \left(-\dfrac{1}{\varphi}\right)^{r+1}\left(-\dfrac{1}{\varphi}\right)^{-1} \\
	&= \varphi^{r+1}(\varphi^{-2} + \varphi^{-1}) + \left(-\frac{1}{\varphi}\right)^{r+1} \left[ \left(-\frac{1}{\varphi}\right)^{-2} + \left(-\frac{1}{\varphi}\right)^{-1}\right] \\
	&= \varphi^{r+1} \left(\frac{1}{\varphi^2} + \frac{1}{\varphi}\right) + \left(-\frac{1}{\varphi}\right)^{r+1}(\varphi^2-\varphi) \\
	&= \varphi^{r+1} \cdot \frac{1+\varphi}{\varphi^2}+\left(-\frac{1}{\varphi}\right)^{r+1}(\varphi^2-\varphi).
\end{align*}
Since $\varphi$ is the root of $m^2 -m -1 =0$, we obtain $\varphi^2-\varphi-1=0$, then $\varphi^2 = \varphi + 1$, thus
 \begin{align*}
 M_{r} + M_{r+2} &= \varphi^{r+1} \cdot \frac{1+\varphi}{\varphi + 1}+\left(-\frac{1}{\varphi}\right)^{r+1}(\varphi + 1-\varphi)\\
 &= \varphi^{r+1}+\left(-\frac{1}{\varphi}\right)^{r+1}.
 \end{align*}	
 
 Thus $P(r+1)$ is true.
 
By principle of strong mathematical induction, we obtain 
\begin{center}
$\varphi^n + \left(-\dfrac{1}{\varphi}\right)^n = M_{n-1} + M_{n+1} $, for all $n \in \mathbb{N}$.
\end{center}
\end{enumerate}
\end{proof}

\par
\noindent We will consider $ k = 2 $ in equation \eqref{recursive metallic} and \eqref{metallic ratio order k sequence}. Then the recurrence relation would become 
\begin{equation}\label{k = 2 in reccurence}
	 M_{n+2} = 2M_{n+1} + M_n.
	 \textnormal{ for } n \geq 1
\end{equation}  
where $M_0=0, M_1=1, M_2=2.$
\newline

From equation \eqref{metallic ratio order k sequence} we get sequence whose terms are given by 
\begin{center}
	$0,1,2,5,12,29,70,169,408,985,2378,\ldots $ 
\end{center}

It can be show that (see \cite{R20}) the ratio of successive terms of this sequence approaches the silver ratio
$ \rho_{2} = \lambda = 1 + \sqrt{2} $. We will now prove the following theorem.
\begin{theorem}\label{silver ratio theorem}
If $\bigl\{ M_n \bigr\}$ is the sequence as defined on equation \eqref{k = 2 in reccurence}, then for any positive integer $ n $, we have
\begin{equation}\label{silver ratio recursive}
	 \lambda^n + \left(-\frac{1}{\lambda}\right)^n = M_{n-1} + M_{n+1}.  
\end{equation}
\end{theorem}

\begin{proof}
We prove by using principle of strong mathematical induction on $n \in \mathbb{N}.$

\quad Let $P(n):\lambda^n + \left(-\dfrac{1}{\lambda}\right)^n = M_{n-1} + M_{n+1}$. 
\begin{enumerate}[label=(\roman*),leftmargin = 1.5cm]
	\item To show  that $P(1)$ is true, that is 
	$$\lambda - \dfrac{1}{\lambda} = M_{0} + M_{2}.$$

 From Lemma \ref{lemma} and $k=2$, we have $\rho_2 - \dfrac{1}{\rho_2} =2. $
 
 Since $ \lambda = \rho_2$ and $M_0=0, M_2=2,$ then 
  \begin{align*}
 	\lambda - \dfrac{1}{\lambda} &= 2\\
 	&= 0 + 2\\
 	&= M_{0} + M_{2},
 \end{align*}
 thus  $P(1)$  is true.
 
 \item For $r \geq 1$. We assume that $P(r-1)$ and $P(r)$ are true, that is
 \begin{equation*}
 	\lambda^{r-1} + \left(-\dfrac{1}{\lambda}\right)^{r-1} = M_{r-2} + M_{r}, 
 \end{equation*}
 
 and 
\begin{equation*}
	\lambda^{r} + \left(-\dfrac{1}{\lambda}\right)^{r} = M_{r-1} + M_{r+1},
\end{equation*}

respectively.

To show that $P(r+1$) is true, that is
$$ \lambda^{r+1} + \left(-\dfrac{1}{\lambda}\right)^{r+1}= M_{r} + M_{r+2}. $$  
 Consider
 \begin{align*}
	M_{r} + M_{r+2} &= 2M_{r-1} + M_{r-2} + 2M_{r+1} + M_r \\
	&= M_{r-2} + M_r + 2M_{r-1} + 2M_{r+1} \\
	&= M_{r-2} + M_r + 2(M_{r-1} + M_{r+1}) \\
	&= \lambda^{r-1} + \left(-\dfrac{1}{\lambda}\right)^{r-1} +2\left[\lambda^{r} + \left(-\dfrac{1}{\lambda}\right)^{r} \right]\\
	&= \lambda^{r-1} + \left(-\dfrac{1}{\lambda}\right)^{r-1} +2\lambda^{r} + 2\left(-\dfrac{1}{\lambda}\right)^{r} \\
	&= \lambda^{r-1} + 2\lambda^{r} + \left(-\dfrac{1}{\lambda}\right)^{r-1} + 2\left(-\dfrac{1}{\lambda}\right)^{r} \\
%	&= \varphi^{r+1}\varphi^{-2} + \varphi^{r+1}\varphi^{-1} + \left(-\dfrac{1}{\varphi}\right)^{r+1}\left(-\dfrac{1}{\varphi}\right)^{-2} + \left(-\dfrac{1}{\varphi}\right)^{r+1}\left(-\dfrac{1}{\varphi}\right)^{-1} \\
	&= \lambda^{r+1}(\lambda^{-2} + 2\lambda^{-1}) + \left(-\frac{1}{\lambda}\right)^{r+1} \left[ \left(-\frac{1}{\lambda}\right)^{-2} + 2\left(-\frac{1}{\lambda}\right)^{-1}\right] \\
	&= \lambda^{r+1} \left(\frac{1}{\lambda^2} + \frac{2}{\lambda}\right) + \left(-\frac{1}{\lambda}\right)^{r+1}(\lambda^2-2\lambda) \\
	&= \lambda^{r+1} \cdot \frac{1+2\lambda}{\lambda^2}+\left(-\frac{1}{\lambda}\right)^{r+1}(\lambda^2-2\lambda).
\end{align*}

Since $\lambda$ is the root of $m^2 -2m -1 =0$, we obtain $\lambda^2-2\lambda-1=0$, then $\lambda^2 = 2\lambda + 1$, thus

 \begin{align*}
 M_{r} + M_{r+2} &= \lambda^{r+1} \cdot \frac{1+2\lambda}{2\lambda+1}+\left(-\frac{1}{\lambda}\right)^{r+1}(2\lambda+1-2\lambda)\\
 &= \lambda^{r+1}+\left(-\frac{1}{\lambda}\right)^{r+1}.
 \end{align*}	
 
 
 Thus $P(r+1)$ is true.

By principle of strong mathematical induction, we obtain

\begin{center}
$\lambda^n + \left(-\dfrac{1}{\lambda}\right)^n = M_{n-1} + M_{n+1} $, for all $n \in \mathbb{N}$.
\end{center}
\end{enumerate}
\end{proof}
\noindent We will consider $ k = 3 $ in equation \eqref{recursive metallic} and \eqref{metallic ratio order k sequence}. Then the recurrence relation would become 
\begin{equation}\label{k = 3 in reccurence}
	 M_{n+2} = 3M_{n+1} + M_n.
	 \textnormal{ for } n \geq 1
\end{equation}  
where $M_0=0, M_1=1, M_2=3.$
\newline

From equation \eqref{metallic ratio order k sequence} we get sequence whose terms are given by 
\begin{center}
	$0,1,3,10,33,109,360,1189,3927,12970,\ldots  $ 
\end{center}

It can be show that (see \cite{R20}) the ratio of successive terms of this sequence approaches the silver ratio
$\rho_{3} = \mu = \dfrac{3 + \sqrt{13}}{2}$. We will now prove the following theorem.
\begin{theorem}\label{bronze ratio theorem}
If $\bigl\{ M_n \bigr\}$ is the sequence as defined on equation \eqref{k = 3 in reccurence},  then for any positive integer $ n $, we have
\begin{equation}\label{silver ratio recursive}
	 \mu^n + \left(-\frac{1}{\mu}\right)^n = M_{n-1} + M_{n+1}.  \textnormal{ for } n \geq 1
\end{equation}
\end{theorem}

\begin{proof}
We prove by using principle of mathematical induction on $n \in \mathbb{N}.$

\quad Let $P(n):\mu^n + \left(-\dfrac{1}{\mu}\right)^n = M_{n-1} + M_{n+1}$. 
\begin{enumerate}[label=(\roman*),leftmargin = 1.5cm]
	\item To show  that $P(1)$ is true, that is 
	$$\mu - \dfrac{1}{\mu} = M_{0} + M_{2}.$$ 

 From Lemma \ref{lemma} and $k=3$, we have $\rho_3 - \dfrac{1}{\rho_3} =3 .$
 
 Since $ \mu = \rho_3$ and $M_0=0, M_2=3,$ then 
  \begin{align*}
 	\mu - \dfrac{1}{\mu} &= 3\\
 	&= 0 + 3\\
 	&= M_{0} + M_{2},
 \end{align*}
 thus  $P(1)$  is true.
 
 \item For $r \geq 1$. We assume that $P(r-1)$ and $P(r)$ are true, that is
 \begin{equation*}
	\mu^{r-1} + \left(-\dfrac{1}{\mu}\right)^{r-1} = M_{r-2} + M_{r} ,
\end{equation*}

and
\begin{equation*}
	\mu^{r} + \left(-\dfrac{1}{\mu}\right)^{r} = M_{r-1} + M_{r+1},
\end{equation*}

repectively.

To show that $P(r+1)$ is true, that is
$$ \mu^{r+1} + \left(-\dfrac{1}{\mu}\right)^{r+1}= M_{r} + M_{r+2}. $$  
 Consider
 \begin{align*}
	M_{r} + M_{r+2} &= 3M_{r-1} + M_{r-2} + 3M_{r+1} + M_r \\
	&= M_{r-2} + M_r + 3M_{r-1} + 3M_{r+1} \\
	&= M_{r-2} + M_r + 3(M_{r-1} + M_{r+1}) \\
	&= \mu^{r-1} + \left(-\dfrac{1}{\mu}\right)^{r-1} +3\left[\mu^{r} + \left(-\dfrac{1}{\mu}\right)^{r} \right]\\
	&= \mu^{r-1} + \left(-\dfrac{1}{\mu}\right)^{r-1} +3\mu^{r} + 3\left(-\dfrac{1}{\mu}\right)^{r} \\
	&= \mu^{r-1} + 3\mu^{r} + \left(-\dfrac{1}{\mu}\right)^{r-1} + 3\left(-\dfrac{1}{\mu}\right)^{r} \\
%	&= \varphi^{r+1}\varphi^{-2} + \varphi^{r+1}\varphi^{-1} + \left(-\dfrac{1}{\varphi}\right)^{r+1}\left(-\dfrac{1}{\varphi}\right)^{-2} + \left(-\dfrac{1}{\varphi}\right)^{r+1}\left(-\dfrac{1}{\varphi}\right)^{-1} \\
	&= \mu^{r+1}(\mu^{-2} + 3\mu^{-1}) + \left(-\frac{1}{\mu}\right)^{r+1} \left[ \left(-\frac{1}{\mu}\right)^{-2} + 3\left(-\frac{1}{\mu}\right)^{-1}\right] \\
	&= \mu^{r+1} \left(\frac{1}{\mu^2} + \frac{3}{\mu}\right) + \left(-\frac{1}{\mu}\right)^{r+1}(\mu^2-3\mu) \\
	&= \mu^{r+1} \cdot \frac{1+3\mu}{\mu^2}+\left(-\frac{1}{\mu}\right)^{r+1}(\mu^2-3\mu).
\end{align*}
Since $\mu$ is the root of $m^2 -3m -1 =0$, we obtain $\mu^2-3\mu-1=0$, then $\mu^2 = 3\mu + 1$, thus
 \begin{align*}
 M_{r} + M_{r+2} &= \mu^{r+1} \cdot \frac{1+3\mu}{3\mu+1}+\left(-\frac{1}{\mu}\right)^{r+1}(3\mu+1-3\mu)\\
 &= \mu^{r+1}+\left(-\frac{1}{\mu}\right)^{r+1}.
 \end{align*}	
 
 Thus $P(r+1)$ is true.
 
By principle of strong mathematical induction, we obtain 
\begin{center}
$\mu^n + \left(-\dfrac{1}{\mu}\right)^n = M_{n-1} + M_{n+1} $, for all $n \in \mathbb{N}$.
\end{center}
\end{enumerate}
\end{proof}

Next, I will obtain a general result connecting terms of the metallic ratio sequence defined in equation \eqref{recursive metallic} and 
\eqref{metallic ratio order k sequence}. Then the recurrence relation of general metallic ratio sequence is given by
\begin{equation*}
	 M_{n+2} =kM_{n+1}+M_n.
\end{equation*}  
where $M_0=0, M_1=1, M_2=k.$
\newline

From equation \eqref{metallic ratio order k sequence}, we get a sequence whose terms are given by 
\begin{center}
	$0,1,k,k^2 + 1,k^3 + 2k,k^4 + 3k^2 + k, k^5 + 4k^3 + k^2 + 2k, k^6 + 5k^4 +k^3 + 5k^2 + k,\ldots$ 
\end{center}

It can be shown that (see \cite{R20}) the ratio of successive terms of this sequence approaches the metallic ratio of order $k$ given by $\rho_k = \dfrac{k+\sqrt{k^2+4}}{2}$. We 
will now prove the following theorem.
\begin{theorem}\label{metallic ratio theorem}
If $\bigl\{ M_n \bigr\}$ is the sequence as defined on equation \eqref{recursive metallic}, then for any positive integer $ n $, we have
\begin{equation}\label{metallic ratio recursive}
	 \rho_k ^n + \left(-\frac{1}{\rho_k }\right)^n = M_{n-1} + M_{n+1}. 
\end{equation}
\end{theorem}

\begin{proof}
We prove by using principle of strong mathematical induction on $n \in \mathbb{N}$

\quad Let $P(n): \rho_k ^n + \left(-\dfrac{1}{\rho_k }\right)^n = M_{n-1} + M_{n+1}.$ 
\begin{enumerate}[label=(\roman*),leftmargin = 1.5cm]
	\item To show  that $P(1)$ is true, that is 
	$$\rho_k - \dfrac{1}{\rho_k} = M_{0} + M_{2}.$$ 

 From Lemma \ref{lemma}, we have $\rho_k - \dfrac{1}{\rho_k} =k $ and $M_0=0, M_2=k,$ then 
  \begin{align*}
 	\rho_k - \dfrac{1}{\rho_k} &= k\\
 	&= 0 + k\\
 	&= M_{0} + M_{2},
 \end{align*}
 thus  $P(1)$  is true.
 
 \item For $r \geq 1$. We assume that $P(r-1)$ and $P(r)$ are true, that is
 \begin{equation*}
	\rho_k^{r-1} + \left(-\dfrac{1}{\rho_k}\right)^{r-1} = M_{r-2} + M_{r}, 
\end{equation*}

and
\begin{equation*}
	\rho_k^{r} + \left(-\dfrac{1}{\rho_k}\right)^{r} = M_{r-1} + M_{r+1},
\end{equation*}
	
respectively.

To show that $P(r+1$) is true, that is
\begin{equation*}
	 \rho_k^{r+1} + \left(-\dfrac{1}{\rho_k}\right)^{r+1} = M_{r} + M_{r+2}.
\end{equation*}  
 Consider
 \begin{align*}
	M_{r} + M_{r+2} &= kM_{r-1} + M_{r-2} + kM_{r+1} + M_r \\
	&= M_{r-2} + M_r + kM_{r-1} + kM_{r+1} \\
	&= M_{r-2} + M_r + k(M_{r-1} + M_{r+1}) \\
	&= \rho_k^{r-1} + \left(-\dfrac{1}{\rho_k}\right)^{r-1} +k\left[\rho_k^{r} + \left(-\dfrac{1}{\rho_k}\right)^{r} \right]\\
	&= \rho_k^{r-1} + \left(-\dfrac{1}{\rho_k}\right)^{r-1} +k\rho_k^{r} + k\left(-\dfrac{1}{\rho_k}\right)^{r} \\
	&= \rho_k^{r-1} + k\rho_k^{r} + \left(-\dfrac{1}{\rho_k}\right)^{r-1} + k\left(-\dfrac{1}{\rho_k}\right)^{r} \\
%	&= \varphi^{r+1}\varphi^{-2} + \varphi^{r+1}\varphi^{-1} + \left(-\dfrac{1}{\varphi}\right)^{r+1}\left(-\dfrac{1}{\varphi}\right)^{-2} + \left(-\dfrac{1}{\varphi}\right)^{r+1}\left(-\dfrac{1}{\varphi}\right)^{-1} \\
	&= \rho_k^{r+1}(\rho_k^{-2} + k\rho_k^{-1}) + \left(-\frac{1}{\rho_k}\right)^{r+1} \left[ \left(-\frac{1}{\rho_k}\right)^{-2} + k\left(-\frac{1}{\rho_k}\right)^{-1}\right] \\
	&= \rho_k^{r+1} \left(\frac{1}{\rho_k^2} + \frac{k}{\rho_k}\right) + \left(-\frac{1}{\rho_k}\right)^{r+1}(\rho_k^2-k\rho_k) \\
	&= \rho_k^{r+1} \cdot \frac{1+k\rho_k}{\rho_k^2}+\left(-\frac{1}{\rho_k}\right)^{r+1}(\rho_k^2-k\rho_k).
\end{align*}
Since $\rho_k$ is the root of $m^2 -km -1 =0$, we obtain $\rho_k^2-k\rho_k-1=0$, then $\rho_k^2 = k\rho_k + 1$, thus
 \begin{align*}
 M_{r} + M_{r+2} &= \rho_k^{r+1} \cdot \frac{1+k\rho_k}{k\rho_k+1}+\left(-\frac{1}{\rho_k}\right)^{r+1}(k\rho_k+1-k\rho_k)\\
 &= \rho_k^{r+1} +\left(-\frac{1}{\rho_k}\right)^{r+1}.
 \end{align*}	
 
 Thus $P(r+1)$ is true.
 
By principle of strong mathematical induction, we obtain 
\begin{center}
$\rho_k^n + \left(-\dfrac{1}{\rho_k}\right)^n = M_{n-1} + M_{n+1} $, for all $n \in \mathbb{N}$.
\end{center}
\end{enumerate}
\end{proof}


\section{Conclusion}

Introducing golden, silver and bronze ratios through general metallic ratio sequence as defined in equation \eqref{recursive metallic} and \eqref{metallic ratio order k sequence} 
I had established three interesting theorems \ref{golden ratio theorem}, \ref{silver ratio theorem} and \ref{bronze ratio theorem} respectively. I had 	
established the same result for general metallic ratio of order $k$. It is amusing to notice that all metallic ratios 
satisfy similar relation regarding the integral powers and alternate terms of the sequence. This result may pave 
way for proving other similar results regarding metallic ratios.

%--------------------------

\printbibliography

%--------------------------
	
\end{document}